\pdfminorversion=4

\documentclass{beamer}

\usepackage{beamerthemeshadow}

\usepackage[ngerman]{babel}
\usepackage[utf8]{inputenc}

\usepackage{multirow}
\usepackage{algpseudocode}
\usepackage{varwidth}
\usepackage{amsmath}
\usepackage[autostyle=true]{csquotes}
\usepackage[backend=bibtex8]{biblatex}

\usepackage{pgfplots}

\setbeamertemplate{navigation symbols}{}
\setbeamertemplate{bibliography item}[text]
\useoutertheme{infolines}

\bibliography{./refs}

\begin{document}

\title[Asymptotische Analyse von EFs]{Asymptotische Analyse von Erzeugenden-Funktonen}
\subtitle{mit dem Anwendungsbeispiel regulärer Sprachen}
\author{Daniel Busch}
\date{\today}

\begin{frame}
  \titlepage
\end{frame}

\begin{frame}\frametitle{Inhalt}
  \begin{itemize}
    \item Grundlagen zur Berechnung der Asymptotik von EFs
    \uncover<2-5>{\item Asymptotik von rationalen EFs}
    \uncover<3-5>{\item Asymptotik von meromorphen EFs}
    \uncover<4-5>{\item Rationale EFs und reguläre Sprachen}
    \uncover<5-5>{\item Ein paar Beispiele}
  \end{itemize}
\end{frame}

\begin{frame}\frametitle{Asymptotik rationaler EFs}
  \begin{block}{Satz 1: Asymptotik rationaler EFs bei einer eindeutigen Polstelle kleinsten Modulus (Theorem 4.1, [SF96])}
  	Sei $f = \frac{P}{Q}$ eine rationale Funktion mit $P, Q \in \mathbb{C}[z]$ teilerfremd. Sei weiter $z_0 = \frac{1}{\beta}$ die eindeutige Polstelle, sodass $|\beta| > |\alpha|$ für alle Polstellen $\frac{1}{\alpha} \neq \frac{1}{\beta}$. Dann gilt
  	\[
  	  [z^n] f \sim C \beta^n n^{\nu - 1} \,\, \text{ mit } \,\, C = \nu \frac{(-\beta)^\nu P(\frac{1}{\beta})}{\frac{d^\nu}{dz^\nu} Q(\frac{1}{\beta})} \text{.}
  	\]
  \end{block}
\end{frame}

\begin{frame}\frametitle{Asymptotik rationaler EFs}
  \textbf{Beweis:}
  Wir benötigen zunächst die Partialbruchzerlegung der Funktion $f$. \\
  \qquad \\
  \begin{block}{Partialbruchzerlegung}
    Sei $f = \frac{P}{Q}$ eine rationale Funktion mit $P, Q \in \mathbb{C}[z]$ und $0 \leq \text{grad}(P) < \text{grad}(Q) = k$. $Q$ habe die Produktdarstellung
    \[
      Q = c (z - \frac{1}{\alpha_1})^{\nu_1} \cdots (z - \frac{1}{\alpha_k})^{\nu_k} \text{.}
    \]
    Dann besitzt $f$ eine Darstellung der Form
    \[
      \frac{P}{Q} = \sum\limits_{i = 0}^k \sum\limits_{j = 0}^{\nu_i} \frac{a_{ij}}{(z - \frac{1}{\alpha_i})^j} \text{.}
    \]
  \end{block}
\end{frame}

\begin{frame}\frametitle{Asymptotik rationaler EFs}
  Offensichtlich gehören alle Terme der Summe zu einem der Polstellen der Funktion $f$ und können in die Form $a'_{ij} (1 - \alpha_i z)^{-j}$ gebracht werden:
  \begin{align*}
    \frac{a_{ij}}{(z - \frac{1}{\alpha_i})^j} = \frac{(-1)^j a_{ij}}{(\frac{1}{\alpha_i} - z)^j}
    = \frac{(-1)^j \frac{1}{\alpha_i^j} a_{ij}}{(1 - \alpha_i z)^j}
    = \frac{a'_{ij}}{(1 - \alpha_i z)^j}
  \end{align*}
  Da es nach Annahme eine eindeutige Polstelle $\frac{1}{\beta}$ kleinsten Modulus gibt ist klar:
  \[
    |\alpha_i| < |\beta| \quad \forall \alpha_i
  \]
  
  Im Folgenden wird gezeigt, dass alle Terme bis auf den Term $c_0 (1 - \beta z)^{-\nu_\beta}$ asymptotisch vernachlässigbar sind.
\end{frame}

\begin{frame}\frametitle{Asymptotik rationaler EFs}
  Wir betrachten zunächst den $n$-ten Kooeffizienten einer der Terme (der konstante Faktor kann ignoriert werden):
  \begin{align*}
    [z^n] \frac{1}{(1 - \alpha z)^{\nu_\alpha}} &= \frac{1}{n!} \frac{d^n}{dz^n} \frac{1}{(1 - \alpha z)^{\nu_\alpha}} \bigg|_{z =0} \\
    &= \frac{1}{n!} \left( (-\nu_\alpha) (-\nu_\alpha - 1) \cdots (-\nu_\alpha - (n - 1) \right) (-\alpha)^n \\
    &= \frac{(\nu_\alpha + n - 1)!}{(\nu_\alpha - 1)! n!} \alpha^n \\
    &= \begin{pmatrix}
      \nu_\alpha + n - 1 \\
      \nu_\alpha - 1
    \end{pmatrix}
    \alpha^n
  \end{align*}
\end{frame}

\begin{frame}\frametitle{Asymptotik rationaler EFs}
  Für den Binomialkoeffizienten gilt:
  \[
    \begin{pmatrix}
      n \\ k
    \end{pmatrix}
    = \frac{n!}{k! (n - k)!}
    = \frac{n (n - 1) \cdots (n - k + 1)}{k!}
    \leq \frac{n^k}{k!}
  \]
  \[
    \lim\limits_{n \rightarrow \infty} \frac{n (n - 1) \cdots (n - k + 1)}{n^k} = 1 \implies \begin{pmatrix}
      n \\ k
    \end{pmatrix} \sim \frac{n^k}{k!}
  \]
  Damit lassen sich die Koeffizienten mit
  \[
    [z^n] \frac{1}{(1 - \alpha z)^{\nu_\alpha}}
    \sim \frac{(\nu_\alpha + n - 1)^{\nu_\alpha - 1}}{(\nu_\alpha - 1)!} \alpha^n
    \sim \frac{n^{\nu_\alpha - 1}}{(\nu_\alpha - 1)!} \alpha^n
  \]
  asymptotisch abschätzen.
\end{frame}

\begin{frame}\frametitle{Asymptotik rationaler EFs}
  Für $|\alpha| < |\beta|$ gilt nun aber
  \[
    n^k \alpha^n \in o(\beta^n) \text{, }
  \]
  denn es ist
  \begin{align*}
    \lim\limits_{n \rightarrow \infty} \frac{n^k \alpha^n}{\beta^n}
    = \lim\limits_{n \rightarrow \infty} \frac{n^k}{\left( \frac{\beta}{\alpha} \right)^n} = 0 \text{.}
  \end{align*}
  Für die Asymptotik von $f$ gilt dann:
  \[
    [z^n] f \sim [z^n] \frac{c_0}{(1 - \beta z)^{\nu_\beta}}
  \]
\end{frame}

\begin{frame}\frametitle{Asymptotik rationaler EFs}
  Um die Konstante $c_0$ zu bestimmen, kann man einfach das Verhalten der Funktion in der Nähe der Polstelle $1 / \beta$ betrachten:
  \[
    c_0 = \lim\limits_{z \rightarrow 1 / \beta} (1 - \beta z)^{\nu_\beta} f(z) = \lim\limits_{z \rightarrow 1 / \beta} (1 - \beta z)^{\nu_\beta} \frac{P(z)}{Q(z)}
  \]
  Da $P$ und $Q$ teilerfremd sind, folgt mit der $\nu_\beta$-fachen Anwendung der Regel von L’Hospital:
  \[
    c_0 = P(1 / \beta) \lim\limits_{z \rightarrow 1 / \beta} \frac{\frac{d^{\nu_\beta}}{dz^{\nu_\beta}} (1 - \beta z)^{\nu\beta}}{\frac{d^{\nu_\beta}}{dz^{\nu_\beta}} Q(z)} = P(1 / \beta) \frac{\nu_\beta! (-\beta)^{\nu_\beta}}{\frac{d^{\nu_\beta}}{dz^{\nu_\beta}} Q(1 / \beta)}
  \]
\end{frame}

\begin{frame}\frametitle{Asymptotik rationaler EFs}
  Zusammengefasst ergibt
  \[
    [z^n] \frac{c_0}{(1 - \beta z)^{\nu_\beta}}
    \sim \frac{c_0 n^{\nu_\beta - 1}}{(\nu_\beta - 1)!} \beta^n \,\, \text{ und } \,\,
    c_0 = P(1 / \beta) \frac{\nu_\beta! (-\beta)^{\nu_\beta}}{\frac{d^{\nu_\beta}}{dz^{\nu_\beta}} Q(1 / \beta)} \text{, }
  \]
  dass
  \[
    [z^n] f \sim C \beta^n n^{\nu_\beta - 1} \,\, \text{ mit } \,\, C = \nu_\beta \frac{(-\beta)^{\nu_\beta} P(1 / \beta)}{\frac{d^{\nu_\beta}}{dz^{\nu_\beta}} Q(1 / \beta)} \text{.}
  \]
  $\qquad$
  \hfill
  $\square$
\end{frame}

\begin{frame}\frametitle{Asymptotik meromorpher EFs}
  Dieser Satz lässt sich noch für mehrere (endlich viele) Polstellen, die gemeinsam auf dem Konvergenzring einer EF liegen, verallgemeinern! \\
  \qquad \\
  Dazu allerdings erst noch ein paar Definitionen und Sätze: \\
  \qquad \\
  \begin{block}{Definition: Holomorphe Funktion}
  	Eine Funktion $f: U \rightarrow \mathbb{C}$ heißt holomorph in $z_0 \in \mathbb{C}$, falls $f$ in einer Umgebung $U$ von $z_0$ komplex differenzierbar ist. Dann heißt $f$ auch holomorph auf $U$.
  \end{block}
  \begin{block}{Definition: Meromorphe Funktion}
    Eine komplexe Funktion $f$ heißt meromorph, wenn sie bis auf Polstellen holomorph ist.
  \end{block}
\end{frame}

\begin{frame}\frametitle{Asymptotik meromorpher EFs}
  \begin{block}{Definition: Laurent-Reihe}
  	Die Laurent-Reihe von $f: U \rightarrow \mathbb{C}$ um einen Punkt $c \in \mathbb{C}$ ist gegeben durch
  	\[
  	  f(z) = \sum\limits_{n = -\infty}^\infty a_n (z - c)^n
  	  \,\, \text{ mit } \,\,
  	  a_n = \frac{1}{2 \pi i} \oint_\gamma \frac{f(z)}{(z - c)^{n + 1}} dz \text{.}
  	\]
  \end{block}
  \begin{block}{Definition: Hauptteil und Nebenteil}
    Die Laurent-Reihe besteht aus einem Haupt- und einem Nebenteil:
    \[
      f(z) = H(z) + N(z)
    \]
    Diese sind gegeben durch:
    \[
      H(z) = \sum\limits_{n = 1}^\infty a_{-n} z^{-n}
      \,\, \text{ und } \,\,
      N(z) = \sum\limits_{n = 0}^\infty a_n z^n
    \]
  \end{block}
\end{frame}

\begin{frame}\frametitle{Asymptotik meromorpher EFs}
  \textbf{Bemerkung 1:} Die Laurent-Reihe um einen isolierten Pol $z_0$ entwickelt hat einen bei $k$ abbrechenden Hauptteil, wobei $\nu$ die Multiplizität der Polstelle ist! Damit gilt:
  \begin{align*}
    H(f; z_0) &= \sum\limits_{n = 1}^\nu \frac{a_{-j}}{(z - z_0)^n} \\
              &= \sum\limits_{n = 1}^\nu \frac{(-1)^n a_{-n}}{(z_0 - z)^n} \\
              &= \sum\limits_{n = 1}^\nu \frac{(-1)^n a_{-n}}{z_0^n} \frac{1}{(1 - \frac{z}{z_0})^n}
  \end{align*}
  ($H(f; z_0)$ soll hier den Hauptteil der Laurent-Entwicklung von $f$ um den Punkt $z_0$ bezeichnen)
\end{frame}

\begin{frame}\frametitle{Asymptotik meromorpher EFs}
  Mit der Potenzreihenentwicklung
  \[
  	\frac{1}{(1 - z)^{j + 1}} = \sum\limits_{i = 0}^\infty \begin{pmatrix} i + j \\ i \end{pmatrix} z^n
  \]
  gilt weiter:
  \begin{align*}
    H(f; z_0) &= \sum\limits_{n = 1}^\nu \frac{(-1)^n a_{-n}}{z_0^n} \sum\limits_{m = 0}^\infty \begin{pmatrix} m + n - 1 \\ m \end{pmatrix} \left( \frac{z}{z_0} \right)^m \\
              &= \sum\limits_{m = 0}^\infty \underbrace{\left( \sum\limits_{n = 1}^\nu \begin{pmatrix} m + n - 1 \\ n - 1 \end{pmatrix} \frac{(-1)^n a_{-n}}{z_0^{n + m}} \right)}_{= a_m} z^m
  \end{align*}
  
  \textbf{Bemerkung 2:} Die Laurent-Reihe um einen analytischen Punkt $z_0$ entwickelt hat $H(f; z_0) = 0$!
\end{frame}

\begin{frame}\frametitle{Asymptotik meromorpher EFs}
  \begin{block}{Koeffizienten-Abschätzung (Theorem 2.4.3, [W92])}
    Sei $f(z) = \sum_{n \geq 0} a_n z^n$ analytisch in einer Umgebung, die den Ursprung enthält. Sei $z_0 \neq 0$ eine Singularität von $f$. Dann gilt
    \[
      \forall \varepsilon > 0 \, \exists n_0 \in \mathbb{N} \, \forall n > n_0: |a_n| < \left( \frac{1}{|z_0|} + \varepsilon \right)^n
    \]
    und für unendliche viele $n$ gilt weiter
    \[
      |a_n| > \left( \frac{1}{|z_0|} - \varepsilon \right)^n \text{.}
    \]
  \end{block}
  %\qquad \\
  %Daraus folgt auch unmittelbar:
  %\[
  %  |a_n| \in O \left( \left(\frac{1}{|z_0|} + \varepsilon \right)^n \right)
  %\]
\end{frame}

\begin{frame}\frametitle{Asymptotik meromorpher EFs}
  \begin{block}{Satz 2: Asymptotik meromorhper EFs bei mehreren Polstellen kleinsten Modulus (Theorem 5.2.1, [W92])}
    Sei $f$ eine meromorphe Funktion und analytisch im Ursprung. Seien $z_0, \dots, z_s$ die Polstellen von $f$ mit $r = |z_0| = \dots = |z_s|$ und $|z_0| < |y|$ für alle anderen Polstellen $y$. Ferner sei $r' > r$ Modulus der Polstellen des nächstgrößeren Modulus. Es gilt dann
    \[
      [z^n] f = [z^n] \left( \sum\limits_{i = 0}^s H(f; z_i) \right) + O\left( \left( \frac{1}{r'} + \varepsilon \right)^n \right) \text{.}
    \]
  \end{block}
\end{frame}

\begin{frame}\frametitle{Asymptotik meromorpher EFs}
  \textbf{Beweis:}
  Die Idee ist, die Polstellen von $f$ auf dem Konvergenzkreis $r$ zu entfernen und das Resultat mit der eben gezeigten oberen Schranke abzuschätzen. Dazu subtrahieren wir den Hauptteil von $f$ um $z_0$ entwickelt von $f$ selbst, den Hauptteil der entstandenen Funktion, diesmal aber um $z_1$ entwickelt, wieder von dieser und so weiter, bis keine Polstellen auf $r$ mehr vorhanden sind. \\
  \qquad \\
  \[
    g = f - H(f; z_0)
  \]
  \[
    g' = g - H(g; z_1) = f - H(f; z_0)) - H(f - H(f; z_0); z_1))
  \]
  \[
    \vdots
  \]
\end{frame}

\begin{frame}\frametitle{Asymptotik meromorpher EFs}
  Den Term $H(f - H(f; z_0); z_1))$ kann man aber umschreiben als:
  \begin{align*}
    H(f - H(f; z_0); z_1)) &= \sum\limits_{n = 1}^\infty a_n (z - z_1)^n \\
                           &= \sum\limits_{n = 1}^\infty \left( \frac{1}{2 \pi i} \oint_\gamma \frac{f - H(f; z_0)}{(z - z_0)^{n + 1}} dz \right) (z - z_1)^n \\
                           &= \sum\limits_{n = 1}^\infty \left( \frac{1}{2 \pi i} \oint_\gamma \frac{f}{(z - z_0)^{n + 1}} dz \right) (z - z_1)^n \\
                           &- \sum\limits_{n = 1}^\infty \left( \frac{1}{2 \pi i} \oint_\gamma \frac{H(f; z_0)}{(z - z_0)^{n + 1}} dz \right) (z - z_1)^n \\
                           &= H(f; z_1) - H(H(f; z_0); z_1)
  \end{align*}
\end{frame}

\begin{frame}\frametitle{Asymptotik meromorpher EFs}
  Da $H(f; z_0)$ allerdings analytisch in $z_1$ ist ($H(f; z_0)$ hat nur eine Polstelle und zwar $z_0$) folgt mit Bemerkung 2:
  \[
    H(f - H(f; z_0); z_1)) = H(f; z_1) - \underbrace{H(H(f; z_0); z_1)}_{= 0} =  H(f; z_1)
  \]
  Induktiv erhält man nun eine Funktion
  \[
    g = f - H(f; z_0) - H(f; z_1) - \dots - H(f; z_s) \text{, }
  \]
  von der bekannt ist, dass keine Polstelle mehr auf $r$ liegt. Die nächste Polstelle liegt nach Annahme erst bei $r'$:
  \[
    \left| [z^n] g \right| \in O \left( \left(\frac{1}{r'} + \varepsilon \right)^n \right)
  \]
\end{frame}

\begin{frame}\frametitle{Asymptotik meromorpher EFs}
  Also folgt:
  \[
    [z^n] f = [z^n] (H(f; z_0) + H(f; z_1) + \dots + H(f; z_s)) + O \left( \left(\frac{1}{r'} + \varepsilon \right)^n \right)
  \]
  $\qquad$
  \hfill
  $\square$
\end{frame}

\begin{frame}\frametitle{Asymptotik meromorpher EFs}
  Falls $r' > 1$ ist (andernfalls kann man den obigen Satz einfach mehrfach anwenden), erhält man für die Asymptotik:
  \begin{align*}
    [z^n] f &\sim \sum\limits_{i = 0}^s \sum\limits_{j = 1}^{\nu_s} \begin{pmatrix} n + j - 1 \\ j - 1 \end{pmatrix} (-1)^j \frac{a_{-j}}{z_i^{j + n}} \\
%    &\sim \sum\limits_{i = 0}^s \sum\limits_{j = 1}^{\nu_s} \frac{(-1)^j n^{j - 1} a_{-j}}{(j - 1)! z_i^{j + n}}
  \end{align*}
  ($z_0, \dots, z_s$ sind die Polstellen von $f$, $k_s$ die Multiplizität der jeweiligen Polstelle)
\end{frame}

\begin{frame}\frametitle{Reguläre Sprachen}
  \begin{block}{Definition: Reguläre Sprache}
  	Eine Sprache  $\mathcal{L}$ ist genau dann regulär, wenn es einen regulären Ausdruck gibt, der sie erzeugt.
  \end{block}
  \qquad \\
  Für einen regulären Ausdruck können wir induktiv wie folgt sehr einfach eine Mengenschreibweise finden (wobei $R_1$ und $R_2$ zwei beliebige reguläre Ausdrücke seien):
  \begin{align*}
    R_1 R_2 &\mapsto \mathcal{L}(R_1) \times \mathcal{L}(R_2) \\
    R_1 | R_2 &\mapsto \mathcal{L}(R_1) \cup \mathcal{L}(R_2) \leftrightarrow \mathcal{L}(R_1) + \mathcal{L}(R_2) \\
    (R_1)^* &\mapsto \bigcup\limits_{n \geq 0} \mathcal{L}^n(R_1) \leftrightarrow \text{SEQ}(\mathcal{L}(R_1))
  \end{align*}
\end{frame}

\begin{frame}\frametitle{Reguläre Sprachen}
  So ist zum Beispiel der reguläre Ausdruck für binäre Wörter, die mit einer 1 beginnen gegeben durch
  \[
    R = 1 (0|1)^* \text{,}
  \]
  die zugehörige Sprache ist also
  \begin{align*}
    \mathcal{L}(R) &= \{ 1 \} \times \text{SEQ}(\{ 0, 1 \}) \\
                   &= \{ 1 \} \times (\{ \varepsilon \} + \{ 0, 1 \} + \{ 0, 1 \}^2 + \dots) \text{.}
  \end{align*}
\end{frame}

\begin{frame}\frametitle{Reguläre Sprachen}
  Das Ziel ist es nun eine Erzeugenden-Funktion für eine reguläre Sprache zu ermitteln. \\
  \qquad \\
  Die Zählsequenz der aus dieser Konstruktion abgeleiteten Erzeugenden-Funktion zählt Wörter jedoch eventuell mehrfach. So erzeugt
  \[
    R = 1 (0 | 1)^* (0 | 1)^*
  \]
  dieselbe Sprache wie im Beispiel eben, alle Wörter $w \in \mathcal{L}(R)$ können jedoch auf $|w|$ verschiedene Weisen „zustandekommen“.
\end{frame}

\begin{frame}\frametitle{Reguläre Sprachen}
  Es wird also ein Mechanismus benötigt, um einen regulären Ausdruck in eine eindeutige reguläre Spezifikation zu verwandeln. \\
  \qquad \\
  Es ist bekannt, dass wir einen regulären Ausdruck direkt in einen NFA übersetzen können und diesen mittels der Potenzmengenkonstruktion in einen DFA. \\
  \qquad \\
  
  \begin{block}{Satz von Kleene}
    Zu jedem DFA gibt es einen regulären Ausdruck $R$, sodass jedes Wort $w \in \mathcal{L}(R)$ eindeutig aus $R$ hervorgeht.
  \end{block}
\end{frame}

\begin{frame}\frametitle{Reguläre Sprachen}
  Folglich kann jeder reguläre Ausdruck so umgeformt werden, dass er direkt in eine eindeutige reguläre Spezifikation übersetzt werden kann. \\
  \qquad \\
  \begin{block}{Asymptotische Anzahl von Wörtern der Länge $n$}
    Sei $\mathcal{L}$ eine eindeutige reguläre Spezifiation. Dann induziert die Wortlänge $|w|$, $w \in \mathcal{L}$ eine Größenfunktion und die Zählsequenz $(L_n)_{n \geq 0}$.
  \end{block}
\end{frame}

\begin{frame}\frametitle{Reguläre Sprachen (ein Beispiel)}
  Für $R = a (b|c)^* b (c|d)^*$ über dem Alphabet $\Sigma = \{ a, b, c, d \}$ können wir schreiben:
  \[
    \mathcal{L}(R) = \{ a \} \times \text{SEQ}(\{ b \} + \{ c \}) \times \{ b \} \times \text{SEQ}(\{ c \} + \{ d \})
  \]
  \uncover<2-4>{Daraus ergibt sich die Erzeugenden-Funktion
  \[
  	L(z) = z \cdot \frac{1}{1 - 2z} \cdot z \cdot \frac{1}{1 - 2z} = \frac{z^2}{(1 - 4z)^2}
  \]
  }\uncover<3-4>{und offensichtlich ist
  \[
    P(z) = z^2 \,\, \text{, } \,\, Q(z) = (1 - 2z)^2 \,\, \text{ und } \frac{d^2}{dz^2} Q(z) = 8
  \]
  }\uncover<4-4>{sowie
  \[
  	\beta = 2 \,\, \text{ und } \,\, \nu = 2 \text{.}
  \]
  }
\end{frame}

\begin{frame}\frametitle{Reguläre Sprachen (ein Beispiel)}
  Da es sich um eine eindeutige Polstelle handelt, kann der erste Satz angewendet werden, also ist
  \[
    C = 2 \cdot \frac{(-2)^2 f(\frac{1}{2})}{\frac{d^2}{dz^2} Q(\frac{1}{2})} = 2 \cdot \frac{(-2)^2 \cdot (\frac{1}{2})^2}{8} = \frac{1}{4}
  \]
  und damit
  \[
    [z^n] L \sim C 2^n n = 2^{n - 2} n \text{.}
  \]
\end{frame}

\begin{frame}\frametitle{Reguläre Sprachen (ein Beispiel)}
  Mit ein paar kombinatorischen Überlegungen findet man für den exakten Wert $\sum_{i = 0}^{n - 1} 2^{n - 2}$ (rot) und sieht, dass sich der asymptotische Wert (blau) sehr schnell diesem angleicht: \\
  \begin{center}
  \begin{tikzpicture}
    \begin{axis}[
        height=0.55\textwidth,
        width=\textwidth,
        xmin=2,
        xmax=15,
        ymin=0,
        %ymax=2^20,
        %xlabel=LabelX,
        %xtick={1,...,100},
        %ylabel=LabelY,
        ymode=log,
        log basis y={2}]
	  \addplot[domain={0:15}, color=red] {
        (x - 1) * 2^(x - 2)
      };
      \addplot[domain={0:15}, color=blue] {
        x * 2^(x - 2)
      };
    \end{axis}
  \end{tikzpicture}
  \end{center}
\end{frame}

\begin{frame}\frametitle{Reguläre Sprachen (noch ein Beispiel)}
  Der reguläre Ausdruck $R = a (a|bb)^*$ über dem Alphabet $\Sigma = \{ a, b \}$ lässt sich als
  \[
  	\mathcal{L}(R) = \{ a \} \times \text{SEQ}(\{ a \} + \{ bb \})
  \]
  schreiben \uncover<2-3>{und ergibt die Erzeugenden-Funktion
  \[
    L(z) = z \cdot \frac{1}{1 - (z + z^2)} = \frac{z}{1 - z - z^2} \text{.}
  \]
  }\uncover<3-3>{Die Polstellen der Funktion sind gegeben durch:
  \begin{align*}
    & z^2 - z - 1 = 0 \\
    \iff& (z + \frac{1}{2})^2 - \frac{5}{4} = 0 \\
    \iff& z = -\frac{1}{2} \pm \frac{\sqrt{5}}{2}
  \end{align*}
  }
\end{frame}

\begin{frame}\frametitle{Reguläre Sprachen (noch ein Beispiel)}
  Die betragsmäßig kleinere Nullstelle ist $z = -\frac{1}{2} + \frac{\sqrt{5}}{2}$. Damit ist:
  \[
    \beta = \frac{2}{-1 + \sqrt{5}} \,\, \text{ und } \,\, \nu = 1
  \]
  \uncover<2-4>{
  \[
    P(z) = z \text{, } \,\,
    Q(z) = 1 - z - z^2 \,\, \text{ und } \,\,
    \frac{d}{dz} Q(z) = -1 - 2z
  \]
  }\uncover<3-4>{Und weiter:
  \[
    C = \nu \frac{-\beta \frac{1}{\beta}}{-1 - 2 \frac{1}{\beta}} = \frac{1}{1 + 2 \frac{1}{\beta}} = \frac{1}{1 + (-1 + \sqrt{5})} = \frac{1}{\sqrt{5}}
  \]
  }
  \uncover<4-4>{
  \[
    L \sim C \beta^n = \frac{1}{\sqrt{5}} (\frac{2}{-1 + \sqrt{5}})^n = \frac{1}{\sqrt{5}} (\frac{2 (1 + \sqrt{5})}{5 - 1})^n = \frac{1}{\sqrt{5}} \phi^n \sim f_n
  \]
  }
\end{frame}

\begin{frame}
  \begin{center}
    Fragen?
  \end{center}
\end{frame}

\begin{frame}[allowframebreaks]\frametitle{Referenzen}
  \nocite{*}
  \printbibliography
\end{frame}

\end{document}
